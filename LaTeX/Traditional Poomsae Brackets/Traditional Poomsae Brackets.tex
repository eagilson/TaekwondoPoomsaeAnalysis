% Setting up the document class for standard letter paper in landscape orientation
\documentclass{article}
\usepackage{geometry}
\geometry{
	landscape,
	letterpaper,
	margin=.5cm,
	includehead,
	headheight=3.2cm % Accommodates 3cm logo and text
}
\usepackage{graphicx}
\usepackage{fancyhdr} % For headers and footers

% Configuring header style
\pagestyle{fancy}
\renewcommand{\headrulewidth}{0pt} % Remove line below header
% Including TikZ for creating the tournament bracket
\usepackage{tikz}

% Defining number sequences for different n values (original order)
\def\seqN{{
  {1,4,3,2}, % n=1
  {1,8,5,4,3,6,7,2}, % n=2
  {1,16,9,8,5,12,13,4,3,14,11,6,7,10,15,2}, % n=3
  {1,32,17,16,9,24,25,8,5,28,21,12,13,20,29,4,3,30,19,14,11,22,27,6,7,26,23,10,15,18,31,2} % n=4
}}

% Defining the tournament style as a reusable TikZ pic
\tikzset{pics/tournament/.style={code={
    % Retrieving key values for the tournament style
    \def\pv##1{\pgfkeysvalueof{/tikz/tournament/##1}}
    % Applying provided key-value pairs for the tournament
    \tikzset{tournament/.cd,#1}
    % Calculating index for sequence array (n-1 since sequences are 0-indexed in LaTeX)
    \pgfmathtruncatemacro{\seqIndex}{\pv{n}-1}
    % Getting the sequence for the current n
    \def\currentSeq{\seqN[\seqIndex]}
    % Looping over rounds from n down to 0
    \foreach \XX [remember=\XX as \LastXX] in {\pv{n},...,0}
       {
        % Calculating the number of matches in the current round
        \pgfmathtruncatemacro{\mym}{pow(2,\XX)}
        % Checking if this is the initial round
        \ifnum\XX=\pv{n}
            % Drawing initial match lines with sequence numbers
            \foreach \YY in {1,...,\mym}
                 {
                    % Calculating sequence indices for top and bottom lines (top-to-bottom order)
                    \pgfmathtruncatemacro{\topSeqNum}{\currentSeq[2*(\mym-\YY)]}
                    \pgfmathtruncatemacro{\botSeqNum}{\currentSeq[2*(\mym-\YY)+1]}
                    % Drawing the top line and adding sequence number
                    \draw[blue] (-\XX*\pv{d},{(\YY-\mym/2-1/2)*\pv{l}})
                    -- ++ (\pv{d},0);
                    \node[anchor=east] at (-\XX*\pv{d},{(\YY-\mym/2-1/2)*\pv{l}}) {\topSeqNum};
                    % Drawing the bottom line and adding sequence number
                    \draw[red] (-\XX*\pv{d},{(\YY-\mym/2-1)*\pv{l}})
                    -- ++ (\pv{d},0);
                    \node[anchor=east] at (-\XX*\pv{d},{(\YY-\mym/2-1)*\pv{l}}) {\botSeqNum};
                    % Connecting the lines vertically
                    \draw (-\XX*\pv{d}+\pv{d},{(\YY-\mym/2-1/2)*\pv{l}})
                    |- coordinate[pos=0.25] (p-\XX-\YY)
                    (-\XX*\pv{d}+\pv{d},{(\YY-\mym/2-1)*\pv{l}})
                    ;
                 }
        % Drawing lines for subsequent rounds
        \else
            \foreach \YY [evaluate=\YY as \ZZ using {int(2*\YY)}] in {1,...,\mym}
                 {
                    % Drawing the top line in blue
                    \draw[blue] (p-\LastXX-\ZZ) -- ++ (\pv{d},0);
                    % Connecting the lines vertically in black
                    \draw[black] (p-\LastXX-\ZZ) ++ (\pv{d},0)
                    |- coordinate[pos=0.25] (p-\XX-\YY)
                    (p-\LastXX-\the\numexpr\ZZ-1);
                    % Drawing the bottom line in red
                    \draw[red] (p-\LastXX-\the\numexpr\ZZ-1) -- ++ (\pv{d},0);
                 }
        \fi
       }
    % Drawing the final horizontal line
    \draw (p-0-1) -- ++ (\pv{d},0);
    % Adding four labeled horizontal lines for names in the bottom right corner
    \node[align=left, anchor=south east] at (8,-8) {
        \begin{tabular}{r@{ }l}
            1) & \rule{5cm}{0.4pt} \\
            2) & \rule{5cm}{0.4pt} \\
            3) & \rule{5cm}{0.4pt} \\
            3) & \rule{5cm}{0.4pt} \\
        \end{tabular}
    };
}},
% Setting default values for the tournament
tournament/.cd,n/.initial=2,l/.initial=1,d/.initial=1}

% Starting the document
\begin{document}

% Header for Round of 32 (n=4)
\chead{
    \vbox to 3.2cm{
        \vfill
        \begin{center}
            \textbf{\Huge Traditional Poomsae Round of 32 Bracket}
        \end{center}
        \vfill
    }
}
% Comment out or replace with correct path if logo.jpg is available
\rhead{
        \vfill
        \includegraphics[height=3cm]{Image/logo.jpg}
        \vfill
    }

% Creating the TikZ picture for the tournament bracket (Round 32, n=4)
\begin{tikzpicture}
    % Scaling the tournament to fit the page width and height
    \begin{scope}[scale=1, transform shape]
        \pic at (2,3.75) {tournament={n=4,d=4,l=1}};
    \end{scope}
\end{tikzpicture}

\newpage

% Header for Round of 16 (n=3)
\chead{
    \vbox to 3.2cm{
        \vfill
        \begin{center}
            \textbf{\Huge Traditional Poomsae Round of 16 Bracket}
        \end{center}
        \vfill
    }
}

% Creating the TikZ picture for the tournament bracket (Round 16, n=3)
\begin{tikzpicture}
    % Scaling the tournament to fit the page width and height
    \begin{scope}[scale=1, transform shape]
        \pic at (2,3.75) {tournament={n=3,d=5,l=2}};
    \end{scope}
\end{tikzpicture}

\newpage

% Header for Round of 8 (n=2)
\chead{
    \vbox to 3.2cm{
        \vfill
        \begin{center}
            \textbf{\Huge Traditional Poomsae Round of 8 Bracket}
        \end{center}
        \vfill
    }
}

% Creating the TikZ picture for the tournament bracket (Round 8, n=2)
\begin{tikzpicture}
    % Scaling the tournament to fit the page width and height
    \begin{scope}[scale=1, transform shape]
        \pic at (2,3.75) {tournament={n=2,d=6,l=2.5}};
    \end{scope}
\end{tikzpicture}

\newpage

% Header for Round of 4 (n=1)
\chead{
    \vbox to 3.2cm{
        \vfill
        \begin{center}
            \textbf{\Huge Traditional Poomsae Round of 4 Bracket}
        \end{center}
        \vfill
    }
}

% Creating the TikZ picture for the tournament bracket (Round 4, n=1)
\begin{tikzpicture}
    % Scaling the tournament to fit the page width and height
    \begin{scope}[scale=1, transform shape]
        \pic at (2,3.75) {tournament={n=1,d=7,l=3}};
    \end{scope}
\end{tikzpicture}

% Ending the document
\end{document}